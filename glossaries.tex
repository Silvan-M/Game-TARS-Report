\newglossaryentry{machine learning}{%
  name={machine learning},
  description={A practice which combines computational power with the ability to learn}
}
\newglossaryentry{deep learning}{%
  name={deep learning},
  description={Type of \gls{machine learning} AI which utilises  \glspl{neuronal network}}
}
\newglossaryentry{artificial narrow intelligence}{%
  name={artificial narrow intelligence},
  description={The typical \gls{AI} we encounter everyday, it's very good at a certain task, but can only perform this task}
}
\newglossaryentry{artificial general intelligence}{%
  name={artificial general intelligence},
  description={Type of \gls{AI}, that can perform all tasks a human can perform}
}
\newglossaryentry{artificial super intelligence}{%
  name={artificial super intelligence},
  description={Type of \gls{AI} which is superior to the human brain and can perform all tasks a human is capable of}
}
\newglossaryentry{neuronal network}{%
  name={neuronal network},
  description={Algorithm which mimics the brains capability to learn. Also known as artificial neuronal network (\gls{ANN})}
}
\newglossaryentry{iteration}{%
  name={iteration},
  description={In programming referred to as the amount of cycles a program goes through}
}
\newglossaryentry{ANN}{%
  name={ANN},
  description={Abbreviation for artificial \gls{neuronal network}}
}
\newglossaryentry{neuron}{%
  name={neuron},
  description={A junction, which generates an \gls{output} by a given \gls{input}. The term can refer to a biological junction and its abstracted mathematical model}
}
\newglossaryentry{input}{%
  name={input},
  description={Defines a possibility or act of receiving data which is later processed by a computer}
}
\newglossaryentry{input layer}{%
  name={input layer},
  description={The \gls{layer} of a \gls{neuronal network} where one inputs the information for the AI to train}
}
\newglossaryentry{output layer}{%
  name={output layer},
  description={The \gls{layer} which passes on the final output of the \gls{neuronal network}}
}
\newglossaryentry{activation function}{%
  name={activation function},
  description={A function, which generates the output in a \gls{neuron} by the summed up weighted inputs}
}
\newglossaryentry{TargetNet}{%
  name={TargetNet},
  description={One of the two \glspl{neuronal network} in the \gls{DQN} model. Structurally identical to the \gls{TrainNet}, which get updated every certain step defined by \gls{copy step}. For further details look chapter \ref{sssec:copy_step}}
}
\newglossaryentry{AI}{%
  name={AI},
  description={Abbreviation for \gls{artificial intelligence}}
}
\newglossaryentry{copy step}{%
  name={copy step},
  description={The variable defines in which frequency the \gls{TargetNet} should be updated. For further details read chapter \ref{sssec:copy_step}}
}
\newglossaryentry{artificial intelligence}{%
  name={artificial intelligence},
  description={The  ability  of  a  digital  device  to execute tasks, which are related to human beings and animals.}
}
\newglossaryentry{TrainNet}{%
  name={TrainNet},
  description={One of the two \glspl{neuronal network} in the \gls{DQN} model. The TrainNet decides which action should be taken. For further details look chapter \ref{sssec:copy_step}}
}
\newglossaryentry{layer}{%
  name={neuronal layer},
  description={The structure multiple \glspl{neuron} build up. It has the property that all neurons, which are part of this layer, connect with all other neurons from the previous and following layers}
}

\newglossaryentry{vector}{%
  name={vector},
  description={A mathematical concept which describes the relative position of a point to another. In the machine learning context a vector is a way to condense the information of a layer into a mathematical object}
}
\newglossaryentry{net value}{%
  name={net value},
  description={$[net]$ The value which a \gls{neuron} produces, before it gets forwarded to the activation function}
}
\newglossaryentry{input size}{%
  name={input layer size},
  description={$[n]$ The amount of states which are used to process the incoming data}
}
\newglossaryentry{weight}{%
  name={weight},
  description={$[w]$ A factor which the input gets multiplied with. The weights change to approximate the target values as close as possible}
}
\newglossaryentry{hidden layer}{%
  name={hidden layer},
  description={One of the three \gls{layer} types a \gls{neuronal network} uses. The hidden layer does not have any connections with the in- or output}
}
\newglossaryentry{hidden units}{%
  name={hidden units},
  description={A synonym for \gls{hidden layer}}
}
\newglossaryentry{amount of hidden layers}{%
  name={amount of hidden layers},
  description={$[a]$ The amount which defines how many \glspl{hidden layer} get stacked after each other}
}
\newglossaryentry{output layer size}{%
  name={output layer size},
  description={$[m]$ The amount of states which are used to describe every action the agent can take}
}
\newglossaryentry{scalar product}{%
  name={scalar product},
  description={A scalar product, or also known as dot product. It allocates a numerical value to two \glspl{vector}. This number can be used to determine the angle between the input \glspl{vector}}
}
\newglossaryentry{size}{%
  name={layer size},
  description={$[s]$ Describes the amount of \glspl{neuron} in a \gls{layer}}
}
\newglossaryentry{tensor}{%
  name={tensor},
  description={Tensor is a quantity which can classify scalars, \glspl{vector} or analogue objects in a unified schema. Tensors are often used by the library \lstinline{tensorflow} to make calculations}
}
\newglossaryentry{regression}{%
  name={regression},
  description={A regression is a way to approximate data in a function to determine its tendencies.}
}
\newglossaryentry{back propagation}{%
  name={backpropagation},
  description={Backpropagation computes the gradient of the loss function with respect to the weights of the network. This for a single input–output example. It does so efficiently, unlike a naive direct computation of the gradient, with respect to each weight individually.\cite{Backpropagation}}
}
\newglossaryentry{sigmoid neuron}{%
  name={sigmoid neuron},
  description={A \gls{neuron} with the sigmoid \gls{activation function}. This type of neuron is mostly used in \glspl{hidden layer}}
}
\newglossaryentry{optimization}{%
  name={optimization},
  description={The process of improving the \gls{AI} by changing the \glspl{weight}}
}
\newglossaryentry{output}{%
  name={output},
  description={$[o]$ The final output of the neuron, after being passed to the activation function}
}
\newglossaryentry{convergence}{%
  name={convergence},
  description={In the context of \glspl{neuronal network}, it describes the behaviour of two functions such that the difference of their value get increasingly smaller}
}
\newglossaryentry{divergence}{%
  name={divergence},
  description={In the context of \glspl{neuronal network}, it describes the behaviour of two functions, such that the difference of their value get increasingly larger}
}
\newglossaryentry{gradient}{%
  name={gradient descent},
  description={The gradient describes the slope of multidimensional functions. The negative slope points in the direction with the fastest \gls{convergence}}
}
\newglossaryentry{minimum}{%
  name={minimum},
  description={The point of a function where the lowest y-value is achieved. There are two types of minima. The global minimum yields the lowest value for the whole function. A local minimum however achieves this just in close range}
}
\newglossaryentry{environment}{%
  name={environment},
  description={A model which takes inputs as actions and returns a respective state}
}
\newglossaryentry{reinforcement learning}{%
  name={reinforcement learning},
  description={An AI which tries to learn off of rewards given by the \gls{environment}}
}
\newglossaryentry{linear function}{%
  name={linear function},
  description={An activation function type, which is mostly used for \glspl{input layer}}
}
\newglossaryentry{binary step function}{%
  name={binary step function},
  description={An activation function type, which is mostly used for \glspl{output layer}. The function can only yield the value 0 or 1}
}
\newglossaryentry{ReLU}{%
  name={Rectified linear unit (ReLU)},
  description={An activation function which is used in a neuron. The function is piecewise linear. The output of the function yields 0, until a certain threshold is reached. After this point the function is linear. It is mostly used in \glspl{hidden layer}}
}
\newglossaryentry{target}{%
  name={target},
  description={The target numerically defines what the  \gls{neuronal network} has to strive for. It is used to make learning possible and lead it in a certain direction}
}
\newglossaryentry{learning rate}{%
  name={learning rate},
  description={$[\alpha]$ Factor which defines how drastically weights should change each train step. For further details read chapter \ref{sec:nnmath} or \ref{sssec:alpha}}
}
\newglossaryentry{policy}{%
  name={policy},
  description={Function which describes the optimal strategy}
}
\newglossaryentry{Markov decision process}{%
  name={Markov decision process},
  description={Model which describes decision making, by observing an environment and reacting upon it, with an action}
}
\newglossaryentry{episode}{%
  name={episode},
  description={The machine learning terminology usually referred to as \gls{iteration} in informatics. More specifically defining one round, or in \gls{reinforcement learning} one game}
}
\newglossaryentry{Q-learning}{%
  name={Q-learning},
  description={Utilising a Q-table to determine the most beneficial action}
}
\newglossaryentry{Deep Q-learning}{%
  name={deep Q-learning},
  description={Approximating the Q-table used in \gls{Q-learning} by applying \gls{deep learning}}
}
\newglossaryentry{off-policy}{%
  name={off-policy},
  description={Deep learning algorithm which learns from actions which are located outside of its policy}
}
\newglossaryentry{alpha}{%
  name={alpha},
  description={Synonym for \gls{learning rate}}
}
\newglossaryentry{experience}{%
  name={experience},
  description={Also known as memory. Defines all states, actions, new states and rewards a agent has encountered. For further details read chapter \ref{sssec:copy_step}}
}
\newglossaryentry{memory}{%
  name={memory},
  description={Also known as experience, defines all states, actions, new states and rewards a agent has encountered. For further details look chapter \ref{sssec:copy_step}}
}
\newglossaryentry{reward}{%
  name={reward},
  description={A numerical value which represents a favorability of an action. A negative reward is not favorable, whilst a high reward is favorable}
}
\newglossaryentry{min experience}{%
  name={min experience},
  description={This variable defines what the threshold is, that the \gls{neuronal network} begins to learn. For further details read chapter \ref{sssec:copy_step}}
}
\newglossaryentry{computational power}{%
  name={computational power},
  description={Calculation powers of a computer. Usually refers to the capability of computers, to deal with high amounts of complex calculations}
}
\newglossaryentry{min max algorithm}{%
  name={min max algorithm},
  description={A type of algorithm yielding the optimal decision an agent can take. It calculates all possibilities, to find the most beneficial one}
}
\newglossaryentry{max experience}{%
  name={max experience},
  description={This variable defines how much information should be saved in the \gls{memory}. For further details read chapter \ref{sssec:copy_step}}
}
\newglossaryentry{DQN}{%
  name={DQN},
  description={Abbreviation for \gls{deep Q-learning network}}
}
\newglossaryentry{deep Q-learning network}{%
  name={deep Q-learning network},
  description={A \gls{neuronal network} which uses a \gls{deep learning} method}
}
\newglossaryentry{exploration}{%
  name={exploration},
  description={When the AI takes random action, to discover inexperienced possibilities}
}
\newglossaryentry{exploration rate}{%
  name={exploration rate},
  description={$[\epsilon]$ When the AI takes random action, to discover inexperienced possibilities}
}
\newglossaryentry{exploitation}{%
  name={exploitation},
  description={The act of taking the most beneficial action, also known as the \gls{greedy action}}
}
\newglossaryentry{batch size}{%
  name={batch size},
  description={A variable which defines the amount of previous inputs that get processed at every time step. By increasing this value, one can raise a sense of motion}
}
\newglossaryentry{greedy action}{%
  name={greedy action},
  description={Taking the action the AI considers most beneficial}
}
\newglossaryentry{Q-table}{%
  name={Q-table},
  description={A table consisting of \glspl{Q-value}}
}
\newglossaryentry{Q-value}{%
  name={Q-value},
  description={A number which represents how favorable an action is depending on it's \gls{state}}
}
\newglossaryentry{state}{%
  name={state},
  description={Output of a \gls{environment} describing its current status}
}
\newglossaryentry{CartPole}{%
  name={CartPole},
  description={A prime example of  \gls{machine learning}. The task is, to balance a pole on a cart, which can be moved along a horizontal rail}
}
\newglossaryentry{low-level}{%
  name={low-level},
  description={Level of programming which is close to the processor}
}
\newglossaryentry{one-hot encoding}{%
  name={one-hot encoding},
  description={A method to quantify categorical data. It produces a vector with the length of the data set and 0 and 1 as elements. For further details look chapter \ref{ssec:DQN}}
}
\newglossaryentry{gamma}{%
  name={gamma},
  description={$[\gamma]$ Gamma determines the importance of future rewards by acting as a factor from 0 to 1. For further details read chapter \ref{sssec:gamma}}
}
\newglossaryentry{headless}{%
  name={headless},
  description={The practice of running a program without a graphical user interface}
}
\newglossaryentry{matplotlib}{%
  name={matplotlib},
  description={A library for creating static, animated and interactive visualizations in Python}
}